\documentclass{article}

\usepackage{geometry, xcolor}
\usepackage{titling, titlesec, enumitem, natbib}
\usepackage{hyperref, cite, url, microtype, babel}
\usepackage{mdframed}
\usepackage[normalem]{ulem}

\setlength{\droptitle}{-10em}

\newcommand{\subtitle}[1]{
  \posttitle{
    \par\end{center}
    \begin{center}\normalsize#1\end{center}
    \vskip0.5em}}
\titleformat{\section}
  {\large\bfseries}{\thesection}{1em}{}
\titleformat{\subsection}
  {\normalsize\bfseries}{\thesubsection}{1em}{}
\titleformat{\subsubsection}
  {\normalsize\itshape}{\thesubsubsection}{1em}{}
\titlespacing*{\section}{0pt}{2ex}{1ex}
\titlespacing*{\subsection}{0pt}{1ex}{0.5ex}
\titlespacing*{\subsubsection}{0pt}{0.5ex}{0.25ex}
%\setlist{nosep}
%\setlength{\bibsep}{0.3em}
\setitemize{itemsep=0.25em,topsep=0.5em,parsep=0pt,partopsep=0pt}

\newenvironment{relatedwork}
   {
     \begin{mdframed}[
       leftmargin=1cm,
       rightmargin=0cm,
       innerleftmargin=10pt,
       innerrightmargin=0pt,
       innertopmargin=0.5em,
       innerbottommargin=0.5em,
       linewidth=1pt,
       linecolor=gray,
       topline=false,
       bottomline=false,
       rightline=false
     ]
     \footnotesize
   }
   {
     \end{mdframed}
   }

\newcommand{\disown}[1]{\color{lightgray}\sout{#1}}
\DeclareTextFontCommand{\archive}{\color{lightgray}}
\definecolor{linky}{RGB}{225,255,255}\let\hrefori\href\renewcommand{\href}[2]{{\setlength{\fboxsep}{1pt}\colorbox{linky}{\hrefori{#1}{#2}}}}
\urlstyle{same}

\title{\Large\bf Abhimanyu Pallavi Sudhir}
\subtitle{AI researcher working on program markets in the context of AI and bounded rationality. \\ %\href{mailto:abhimanyupallavisudhir@gmail.com}{abhimanyupallavisudhir@gmail.com} / +44-7771824896 / 55 Leinster Square, London -- W2 4PW 
\vspace{-2em}}
\predate{}\postdate{}\date{}
\preauthor{}\postauthor{}\author{}

\begin{document}

\begingroup
\let\center\flushleft
\let\endcenter\endflushleft
\maketitle
\endgroup

% \section*{Formal}

\section*{Formal education}

\begin{itemize}

    \item University of Warwick $\cdot$ PhD Computer Science $\cdot$ 2022-26 -- supervisor: Long-Tran-Thanh %; Bounded rationality and market dynamics

    \item Imperial College London $\cdot$ MSci Mathematics $\cdot$ 2018-22 -- 1st class honors

\end{itemize}

\section*{Internships}

\begin{itemize}

    \item Goldman Sachs $\cdot$ AI Research Intern $\cdot$ Jan-Aug 2021, London -- Developed and implemented novel methods in NLP and recurrent neural networks for financial forecasting

\end{itemize}

\section*{Research}

\subsection*{Markets and AI (PhD-related)}

My current work focuses on (a) developing  a good enough algorithmic model of market dynamics in order to design AI agents with a market-based structure (b) developing prediction market mechanisms for latent space variables in order to elicit latent knowledge from AIs.

\begin{itemize}

    \item
          Abhimanyu Pallavi Sudhir and Long-Tran Thanh (2024), ``Betting on what is neither verifiable nor falsifiable'', \href{https://arxiv.org/abs/2402.14021}{arxiv.org/abs/2402.14021}

    \item
          Abhimanyu Pallavi Sudhir (2021),
          ``A mathematical definition of property rights in a Debreu economy'',
          \href{https://arxiv.org/abs/2107.09651}{arxiv.org/abs/2107.09651}

\end{itemize}

\begin{relatedwork}
    \emph{Related write-ups and talks.}
    \begin{itemize}[label=—]
        \item LessWrong post (2023): \href{https://www.lesswrong.com/posts/id84oe3LxdzoqinKA/betting-on-what-is-un-falsifiable-and-un-verifiable}{``Betting on what is un-falsifiable and un-verifiable''}
        \item Poster at the Co-operative AI Foundation (CAIF) summer workshop, 2023:

              \href{https://abhimanyu.io/legacy_writing/PhD_presentations/caif.pdf}{abhimanyu.io/legacy\_writing/PhD\_presentations/caif.pdf}
        \item LessWrong post (2022): \href{https://www.lesswrong.com/posts/xqxXrAohXSD3akYCg/meaningful-things-are-those-the-universe-possesses-a}{``Meaningful things are those the universe possesses a semantics for''}
    \end{itemize}
\end{relatedwork}

\subsection*{Consistency checks and forecasting (Berkeley SPAR -- spring 2024)}

Ongoing collaboration with Daniel Paleka and others to develop a consistency benchmark for LLM forecasters, as part of the Berkeley Supervised Program for Alignment Research (Spring 2024).

\subsection*{General mathematics}

Math publications from high-school days; no longer relevant to my work.

\begin{itemize}

    \item
          Abhimanyu Pallavi Sudhir (2019),
          ``Infinitesimal translations and a multivariate Gr\"unwald-Letnikov calculus'', \href{https://arxiv.org/abs/1904.02710}{arxiv.org/abs/1904.02710}

    \item
          Abhimanyu Pallavi Sudhir (2019),
          ``Generalisations of the determinant to interdimensional transformations: a review,'' \href{https://arxiv.org/abs/1904.08097}{arxiv.org/abs/1904.08097}

    \item
          Abhimanyu Pallavi Sudhir (2018),
          ``The generalized Cauchy derivative as a principal value of the Gr\"unwald-Letnikov fractional derivative for divergent expansions,'' \href{https://arxiv.org/abs/1809.08051}{arxiv.org/abs/1809.08051}

    \item
          Abhimanyu Pallavi Sudhir (2014),
          ``On the Determinant-like function and the Vector Determinant,''
          \emph{Advances in Applied Clifford Algebras} (24-3: 805-807), \href{https://link.springer.com/article/10.1007/s00006-014-0455-3}{doi:10.1007/s00006-014-0455-3}

    \item
          Abhimanyu Pallavi Sudhir (2013),
          ``Defining the Determinant-like function for m by n matrices using the exterior algebra,''
          \emph{Advances in Applied Clifford Algebras} (23-4: 787-792),
          \href{https://link.springer.com/article/10.1007/s00006-013-0416-2}{doi:10.1007/s00006-013-0416-2}

\end{itemize}

\section*{Academic service}

\begin{itemize}

    \item \emph{Teaching Assistant for CS141: Functional Programming (Warwick)} $\cdot$ 2023

    \item \emph{Reviewer for Advances in Applied Clifford Algebras (Springer)} $\cdot$ 2020

\end{itemize}

\section*{Courses and workshops attended}

\begin{itemize}

    \item \emph{Co-operative AI Foundation} $\cdot$ Jul 2023 $\cdot$ workshop on AI and cooperative game theory

\end{itemize}

\section*{Other projects}

\subsection*{Equivariant learning (2021-22)}

Final-year MSci project with Professor Jeroen Lamb at Imperial College London exploring equivariant learning and causal DAGs.

Report: \href{https://abhimanyu.io/legacy_writing/Imperial_reports/m4r.pdf}{abhimanyu.io/legacy\_writing/Imperial\_reports/m4r.pdf}

\subsection*{Lie theory (2019)}

Undergraduate research project with Professor Richard Thomas at Imperial College London on Lie groups and algebras.

Report: \href{https://abhimanyu.io/legacy_writing/Imperial_reports/urop.pdf}{abhimanyu.io/legacy\_writing/Imperial\_reports/urop.pdf}

Presentation: \href{https://abhimanyu.io/legacy_writing/Imperial_presentations/lie_theory.pdf}{abhimanyu.io/legacy\_writing/Imperial\_presentations/lie\_theory.pdf}

% \begin{relatedwork}
%     \emph{Related write-ups and talks.}
%     \begin{itemize}[label=—]
% \item Warwick-Imperial Autumn Meeting (Mar 2022) [cancelled due to COVID-19 lockdowns]
%         % \item Imperial Undergraduate Colloquium (Oct 2019)
% \item Imperial 3-minute thesis competition (Oct 2019)
%     \end{itemize}
% \end{relatedwork}


\subsection*{Lean (2018-19)}

Computerized formal proving in Lean with Kevin Buzzard at Imperial College London.

\begin{itemize}
    \item Wrote the \href{https://github.com/leanprover-community/mathlib4/blob/3a7e6bb77ec51d8009107923a4c071a9473ecc5c/Mathlib/Order/Filter/FilterProduct.lean}{FilterProduct.lean} and \href{https://github.com/leanprover-community/mathlib4/blob/3a7e6bb77ec51d8009107923a4c071a9473ecc5c/Mathlib/Data/Real/Hyperreal.lean}{Hyperreal.lean} modules for the Lean math library
          %\item Imperial first-year project poster: \href{https://abhimanyu.io/legacy_writing/Imperial_reports/m1r.pdf}{abhimanyu.io/legacy_writing/Imperial_reports/m1r.pdf}
    \item Formalized the first-year ``Foundations of Analysis'' module exam Blog post:

          \href{https://xenaproject.wordpress.com/2019/05/06/m1f-imperial-undergraduates-and-lean/}{xenaproject.wordpress.com/2019/05/06/m1f-imperial-undergraduates-and-lean/}
\end{itemize}

\subsection*{PhysicsOverflow (2014-15)}

Co-founded \href{https://physicsoverflow.org}{PhysicsOverflow}, a postgraduate-level physics Q\&A site and open peer review system. See \href{https://en.wikipedia.org/wiki/PhysicsOverflow}{en.wikipedia.org/wiki/PhysicsOverflow} for more details.

\begin{itemize}

    \item
          Abhimanyu Pallavi Sudhir and Rahel Knoepfel (2015),
          ``PhysicsOverflow: A postgraduate-level physics Q\&A site and open peer review system,''
          \emph{Asia-Pacific Physics Newsletter} (4-1: 53-55),
          \href{https://dx.doi.org/10.1142/S2251158X15000193}{doi:10.1142/S2251158X15000193}

\end{itemize}


\subsection*{Links}

\begin{itemize}
    \item Email: \href{mailto:abhimanyupallavisudhir@gmail.com}{abhimanyupallavisudhir@gmail.com}
    \item Contact: \href{tel:+44-7771824896}{+44-7771824896}
    \item Website: \href{https://abhimanyu.io/}{abhimanyu.io}
    \item Blog: \href{https://thewindingnumber.blogspot.com/}{TheWindingNumber.blogspot.com}
    \item Google Scholar: \href{https://scholar.google.com/citations?user=lb38BjYAAAAJ&hl=en}{scholar.google.com/citations?user=lb38BjYAAAAJ}
    \item Github: \href{https://github.com/abhimanyupallavisudhir}{github.com/abhimanyupallavisudhir}
    \item LessWrong: \href{https://www.lesswrong.com/users/abhimanyu-pallavi-sudhir}{lesswrong.com/users/abhimanyu-pallavi-sudhir}
\end{itemize}

\end{document}