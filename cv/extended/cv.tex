\documentclass{article}

\usepackage{geometry, xcolor}
\usepackage{titling, titlesec, enumitem, natbib}
\usepackage{hyperref, cite, microtype, babel}
\usepackage[normalem]{ulem}

\setlength{\droptitle}{-10em}
\newcommand{\subtitle}[1]{
  \posttitle{
    \par\end{center}
    \begin{center}\normalsize#1\end{center}
    \vskip0.5em}}
\titleformat{\section}
  {\large\bfseries}{\thesection}{1em}{}
\titleformat{\subsection}
  {\normalsize\bfseries}{\thesubsection}{1em}{}
\titleformat{\subsubsection}
  {\normalsize\itshape}{\thesubsubsection}{1em}{}
\titlespacing*{\section}{0pt}{2ex}{1ex}
\titlespacing*{\subsection}{0pt}{1ex}{0.5ex}
\titlespacing*{\subsubsection}{0pt}{0.5ex}{0.25ex}
\setlist{nosep}
\setlength{\bibsep}{0.3em}

\newcommand{\disown}[1]{\sout{#1}}
\newcommand{\archive}{\color{lightgray}}

\newenvironment{anticv}
{ \begin{itemize}
    \setlength{\itemsep}{0.5em}
    \setlength{\parskip}{0pt}
    \setlength{\parsep}{0pt}     }
{ \end{itemize}                  } 

\title{\Large\bf Abhimanyu Pallavi Sudhir}
\subtitle{AI and game theory researcher. \\ %\href{mailto:abhimanyupallavisudhir@gmail.com}{abhimanyupallavisudhir@gmail.com} / +44-7771824896 / 55 Leinster Square, London -- W2 4PW 
\vspace{-2em}}
\predate{}\postdate{}\date{}
\preauthor{}\postauthor{}\author{}

\begin{document}

\begingroup
\let\center\flushleft
\let\endcenter\endflushleft
\maketitle
\endgroup

\section*{Formal}

\subsection*{Education}
\begin{itemize}
    \item University of Warwick $\cdot$ PhD Computer Science $\cdot$ 2022-26 -- supervisor: Long-Tran-Thanh %; Bounded rationality and market dynamics
    \item Imperial College London $\cdot$ MSci Mathematics $\cdot$ 2018-22 -- 1st class honors
{\archive
    \item Dhirubhai Ambani International School, Mumbai $\cdot$ IB Diploma $\cdot$ 2013-18 -- 44/45
    \item NUS High School, Singapore $\cdot$ 2012-13
    \item Bukit View Primary School, Singapore $\cdot$ 2006-11
}
\end{itemize}

\subsection*{Internships}
\begin{itemize}
    \item Goldman Sachs $\cdot$ AI Research Intern $\cdot$ Jan-Aug 2021% -- Developed and implemented novel machine learning algorithms in NLP, data stream representation, Bayesian DAGs.
{\archive
    \item Jane Street $\cdot$ Spring Week $\cdot$ 14-17 Apr 2020 -- [cancelled due to COVID-19 lockdowns]
    \item Schroders $\cdot$ Spring Week $\cdot$ 12-13 Aug 2020 -- [held virtually due to COVID-19 lockdowns]
    \item Jane Street $\cdot$ Fall Insight Day $\cdot$ 30 Oct 2018
}
\end{itemize}

\section*{Publications}

\subsection*{AI and agent science}

\begin{itemize}

\item 
Abhimanyu Pallavi Sudhir (2023), ``Betting on what is neither verifiable nor falsifiable'', \href{https://arxiv.org/abs/2402.14021}{arxiv.org/abs/2402.14021}

\item
Abhimanyu Pallavi Sudhir (2021),
``A mathematical definition of property rights in a Debreu economy'', 
\href{https://arxiv.org/abs/2107.09651}{arxiv.org/abs/2107.09651}

\end{itemize}

{\archive

\subsection*{General mathematics}

\begin{itemize}

\item
Abhimanyu Pallavi Sudhir (2019),
``Infinitesimal translations and a multivariate Gr\"unwald-Letnikov calculus'', \href{https://arxiv.org/abs/1904.02710}{arxiv.org/abs/1904.02710}

\item
Abhimanyu Pallavi Sudhir (2018),
``The generalized Cauchy derivative as a principal value of the Gr\"unwald-Letnikov fractional derivative for divergent expansions,'' \href{https://arxiv.org/abs/1809.08051}{arxiv.org/abs/1809.08051}

\item
Abhimanyu Pallavi Sudhir (2019),
``Generalisations of the determinant to interdimensional transformations: a review,'' \href{https://arxiv.org/abs/1904.08097}{arxiv.org/abs/1904.08097}

\item
\disown{Abhimanyu Pallavi Sudhir (2014), 
``On the Determinant-like function and the Vector Determinant,'' 
\emph{Advances in Applied Clifford Algebras} (24-3: 805-807), \href{https://link.springer.com/article/10.1007/s00006-014-0455-3}{doi:10.1007/s00006-014-0455-3}}

\item
\disown{Abhimanyu Pallavi Sudhir (2014),
``On the Properties of the Determinant-like function,'' 
(presented at International Conferences on Mathematical Sciences, Chennai, July 17-19, 2014).}

\item
\disown{Abhimanyu Pallavi Sudhir (2013),
``Defining the Determinant-like function for m by n matrices using the exterior algebra,''
\emph{Advances in Applied Clifford Algebras} (23-4: 787-792),
\href{https://link.springer.com/article/10.1007/s00006-013-0416-2}{doi:10.1007/s00006-013-0416-2}}

\item
\disown{Abhimanyu Pallavi Sudhir (2012),
``The Representation of Matrices in unit vector notation,''
\emph{Journal of Mathematics Research} (4-4: 86-91),
\href{https://dx.doi.org/10.5539/jmr.v4n4p86}{doi:10.5539/jmr.v4n4p86}}

\item \disown{All of the crank stuff I posted to PhysicsForums as a kid}

\end{itemize}

\subsection*{Miscellaneous}

\begin{itemize}

\item
\archive{Abhimanyu Pallavi Sudhir and Rahel Knoepfel (2015), 
``PhysicsOverflow: A postgraduate-level physics Q\&A site and open peer review system,'' 
\emph{Asia-Pacific Physics Newsletter} (4-1: 53-55),
\href{https://dx.doi.org/10.1142/S2251158X15000193}{doi:10.1142/S2251158X15000193}}

\end{itemize}

}

\section*{Projects}

\subsection*{Pet projects}

\begin{itemize}

    \item \emph{The mind-killer} $\cdot$ 2022-present $\cdot$ Personal non-academic blog \href{https://copypasta.substack.com}{[Substack]}
    \item \emph{The Winding Number} $\cdot$ 2016-present $\cdot$ Personal academic blog; sample articles \href{https://thewindingnumber.blogspot.com/2019/10/sigma-fields-are-venn-diagrams.html}{[1]}\href{https://thewindingnumber.blogspot.com/2022/10/a-crash-course-on-mathematical-logic.html}{[2]}\href{https://thewindingnumber.blogspot.com/2020/08/hacking-evidential-decision-theory.html}{[3]}\href{https://thewindingnumber.blogspot.com/2019/05/whats-with-e-1x-on-smooth-non-analytic.html}{[4]}

{\archive

    \item \emph{Co-founded PhyscisOverflow} $\cdot$ 2014-18 $\cdot$ \href{https://physicsoverflow.org}{[Website]}\href{https://en.wikipedia.org/wiki/PhysicsOverflow}{[Wikipedia]}

    \item \disown{\emph{The Mathematics and Physics Encyclopedia} $\cdot$ 2010-14 $\cdot$ [Book]\href{https://psiepsilon.wikia.com/}{[Wiki]}\href{https://psiepsilon.wordpress.com/}{[Blog]}\href{https://www.youtube.com/user/abhi99ps}{[YT channel]}}

}

\end{itemize}

\subsection*{Academic service}

\begin{itemize}

    \item \emph{Teaching Assistant for CS141: Functional Programming (Warwick)} $\cdot$ 2023
    
    \item \emph{Reviewer for Advances in Applied Clifford Algebras (Springer)} $\cdot$ 2020-present

\end{itemize}


\subsection*{Write-ups and talks}

\begin{itemize}

    \item \emph{Betting on what is not verifiable nor falsifiable} $\cdot$ 2023 $\cdot$ PhD
    
    \begin{itemize}
        \item Annual Report \href{https://drive.google.com/file/d/1qNZjvwNXTKl8KHc1_m-vPnJyWvpT_-yi/view?usp=sharing}{[pdf]}
        
        \item Warwick Postgraduate colloquium (Dec 2023) \& Warwick Cake Talk (Nov 2023) \href{https://drive.google.com/file/d/1qvkmYbPsZ9kVE1WPljcPSpXZq7wphDqt/view?usp=sharing}{[ppt]}
    \end{itemize}

    \item \emph{Bounded rationality and such} $\cdot$ 2022-23 $\cdot$ PhD

    \begin{itemize}
        \item ``Algorithmic information is at the root of all our problems'', Warwick Postgraduate colloquium (Mar 2023) \href{https://drive.google.com/file/d/1auYDhlsq1i0ZhTs3UsfuuUXYViS9UnqS/view?usp=sharing}{[ppt]}
        \item ``Incompleteness theorems and firing philosophers'', Warwick Cake Talk (Feb 2023) \href{https://drive.google.com/file/d/1IrAc8RWr6gsGurXND3baNWEGaEJP6Tjb/view?usp=sharing}{[ppt]}
        \item PhD proposal \href{https://drive.google.com/file/d/1EK045SAo025lGJtmkOOYI7sIzzIQF03K/view?usp=sharing}{[pdf]}
    \end{itemize}

    \item \emph{When does equivariant learning make sense?} $\cdot$ 2021-22 $\cdot$ final-year project with Jeroen Lamb
        
    \item \emph{A mathematical definition of property rights} $\cdot$ 2021 

    \begin{itemize}
        \item Imperial Undergraduate Colloquium (Feb 2022)
        \item Sheffield SIAM-IMA Applied Math Conference (July 2021) \href{https://drive.google.com/file/d/1AjYV2bo33F7Cm6vmzD1PBA_8L3cnMQ0S/view?usp=sharing}{[ppt]}
    \end{itemize}

    \item \emph{Local normal forms of analytical maps near fixed points} $\cdot$ 2020 $\cdot$ group report and presentation
    
    \item \emph{Lie theory: the topology of groups} $\cdot$ 2019 $\cdot$ UROP reading project with Richard Thomas

    \begin{itemize}
        \item Warwick-Imperial Autumn Meeting (Mar 2022) [cancelled due to COVID-19 lockdowns] 
        \item Imperial Undergraduate Colloquium (Oct 2019) \href{https://drive.google.com/file/d/1F5ZrQmRhkpp7dWFbjHreB-4pX4uR80VX/view?usp=drive_link}{[report]}\href{https://drive.google.com/file/d/1BYsFoRl5F6DBN7ARMdc-I1TrIKJgMqgY/view?usp=drive_link}{[ppt]}
        \item Imperial 3-minute thesis competition (Oct 2019)
    \end{itemize}

    \item \emph{Ultraproducts and hyperreals} $\cdot$ 2018-19 $\cdot$ computerized formal proving with Kevin Buzzard

    \begin{itemize}
        \item Files in the Lean math library on Github, $\approx$1500 loc \href{https://github.com/leanprover-community/mathlib/blob/master/src/data/real/hyperreal.lean}{[hyperreal]}\href{https://github.com/leanprover-community/mathlib/blob/master/src/order/filter/filter_product.lean}{[ultraproduct]}\href{https://github.com/leanprover-community/mathlib/blob/master/src/order/filter/germ.lean}{[germ]}
        \item Formalization of college math exams \href{https://xenaproject.wordpress.com/2019/05/06/m1f-imperial-undergraduates-and-lean/}{[announcement post]}
        \item Poster presentation (Jun 2019) \href{https://drive.google.com/file/d/1FAx-c7pPaaKvjInDIQGr_gnhEnofc_vJ/view?usp=drive_link}{[poster]}
    \end{itemize}

{\archive

    \item \emph{Fractional calculus} $\cdot$ 2017-19

    \begin{itemize}
        \item IMA Tomorrow's Mathematicians Today IMA TMT (Feb 2019) \href{https://drive.google.com/file/d/1F7MZwoudGuYQT7NqGVbvqIsA-073VCpj/view?usp=drive_link}{[ppt]}
        \item Imperial Undergraduate Colloquium (Nov 2018)
    \end{itemize}

    \item \disown{\emph{Generalized determinants} $\cdot$ 2012-19}
    \begin{itemize}
        \item \disown{Intel ISEF (May 2015) + precursor rounds}
        \item \disown{International Conference on Mathematical Sciences 2014 (Jul 2014)}

    \end{itemize}

}

\end{itemize}

{\archive

\subsection*{Courses}

\begin{itemize}
    \item \emph{Machine Learning and Applied Statistics} $\cdot$ Jul 2019 $\cdot$ summer course at Imperial College Business School; 7.5 ECTS, score: 97.5\%
\end{itemize}

}

{\archive

\subsection*{Awards}

\begin{itemize}
    \item Scholarships
    \begin{itemize}
        \item Warwick PhD (2022-26) -- departmental full scholarship
        \item ICBS Machine Learning Summer course (2019) -- departmental full scholarship
    \end{itemize}
    \item Conferences and science fairs
    \begin{itemize}
        \item IMA TMT, London (2019) -- among 4 shortlisted for GCHQ prize
        \item Intel ISEF, Pittsburgh (2015) -- AMS Karl Menger Award
        \item \disown{International Conference on Mathematical Sciences 2014 -- Best Paper Award}
        \item IRIS National Science Fair (2014) -- Gold; Amul Top 3; GUJCOST Merit Award
        \item IRIS National Science Fair (2013) -- Silver; Special Physics Prize
    \end{itemize}
    \item Problem-solving and olympiads
    \begin{itemize}
        \item Imperial Mathematics Competition (2019) -- nationwide finalist
        \item IIT Math Olympiad (2017) -- sixth place nationally in India
        \item Regional Mathematical Olympiad (2016) -- Merit
    \end{itemize}
    \item Kid competitions
    \begin{itemize}
        \item 2012 Bukit Panjang High School Mathematics and Science Challenge -- Team 1st
        \item 2012 American Mathematics Contest -- Certificate of Achievement
        \item 2012 Rio Tinto Science Contest -- High Dist
        \item 2011 Singapore Mathematical Olympiad Junior -- Honorable Mention
        \item 2011 Singapore Mathematical Olympiad for Primary Schools -- Gold
        \item 2011 Singapore and ASEAN Schools' Math Olympiad -- Gold
        \item 2011 Anglo-Chinese Young Whizzes' Challenge -- Gold; Team Round -- Team 2nd
        \item 2011 River Valley Math Comp -- Individual 1st; Team 1st; Team round -- 2nd; Platinum
        \item 2011 St. Andrew's Math and Science Comp -- Individual 1st; Team 1st; Team round -- 1st
        \item 2011 Mathematical Olympiad Talent Quest -- Bronze; Team Round -- Team 3rd
        \item 2011 Australian Mathematics Competition -- High Dist
        \item 2011 Rio Tinto Science Contest -- Credit
        \item 2011 UNSW ICAS -- Math/Sci/English (Dist) Computers (Credit)
        \item 2010 NUSHS Singapore Primary Science Olympiad -- Gold
        \item 2010 NUSHS National Math Olympiad of Singapore -- Bronze
        \item 2010 Anglo-Chinese Mathlympics -- Individual 3rd; Gold
        \item 2010 Anglo-Chinese Young Whizzes' Challenge -- Gold
        \item 2010 Singapore and ASEAN Schools' Math Olympiad -- Gold
        \item 2010 Australian Mathematics Competition -- Dist
        \item 2010 UNSW ICAS -- Math (HighDist) Science (Dist) English/Writing/Computers (Credit)
        \item 2009 UNSW ICAS -- Math (HighDist) Science (Dist) English (Credit)
        \item 2009 Australian Mathematics Competition (Dist)
        \item 2008 UNSW ICAS -- Math/Science/English (Dist)
        \item 2008 Australian Mathematics Competition (Credit)
    \end{itemize}
\end{itemize}

}

% Links -- programming work and profiles

\section*{Links}

\begin{itemize}
    \item Contact:  \href{mailto:abhimanyupallavisudhir@gmail.com}{[email]}\href{tel:+44-7771824896}{[phone]} % 55 Leinster Square, London -- W2 4PW
    \item Websites: \href{https://copypasta.substack.com}{[Copypasta.substack]}\href{https://thewindingnumber.blogspot.com/}{[TheWindingNumber.blogspot]}\href{https://abhimanyu.io/}{[Homepage]}{\archive\href{https://physicsoverflow.org/}{[PhysicsOverflow]}}
    \item Profiles: \href{https://math.stackexchange.com/users/78451/abhimanyu-pallavi-sudhir}{[StackExchange]}\href{https://www.lesswrong.com/users/abhimanyu-pallavi-sudhir}{[LessWrong]}\href{https://twitter.com/abhimanyupasu}{[Twitter]}\href{https://www.linkedin.com/in/abhimanyu-pallavi-sudhir/}{[LinkedIn]}\href{https://scholar.google.com/citations?user=lb38BjYAAAAJ&hl=en}{[Scholar]}\href{https://orcid.org/0000-0002-2506-0515}{[ORCID]}{\archive\href{https://github.com/abhimanyupallavisudhir/}{[Github]}\disown{\href{https://www.physicsforums.com/members/dimension10}{[PhysicsForums]}}}
    \item {\archive Random applets and such: \href{https://abhimanyups.shinyapps.io/BayesianInference/}{[RShiny\_Bayesian\_inference]}\href{https://www.khanacademy.org/profile/abhimanyupallavisudhir/projects}{[KhanAcademy\_applets]}\href{https://thewindingnumber.blogspot.com/p/blog.html}{[Misc\_neural\_network\_stuff]}}
\end{itemize}

\vspace{1em}

\textbf{Key:} Regular, {\archive Archived}, {\archive \disown{Disowned}}

% \vspace{1em}
% \hrulefill
% \vspace{1em}

% \begin{center}
%     {\Large\bf ANTI-CV}
% \end{center}

% \vspace{0.5em}

% Dirt you can dig up on me on the internet:

% \vspace{0.5em}

% \begin{anticv}
%     \item (2012-15) Unoriginal math papers that should not have been published (marked as ``disowned'' above)
%     \item (2010-11) Crackpot writings on PhysicsForums and Researchgate, e.g. trying to ``disprove'' the existence of the numbers 4, 5 and 7.
%     \item (2014-15) Idiotic politicking on Stack Exchange and PhysicsOverflow, a painting from the French revolution with my face photoshopped onto Robespierre's, Wikipedia vandalism, and hundreds of pages of debate on the merits of Windows Mobile 6.5.5 on WinBeta.
%     \item (?-2014) Promotion of inane lefty political ideologies I no longer endorse, including: environmentalism, Dravidian nationalism, Ludditery, radical egalitarianism, animal liberation, anti-humanism, sympathies to communism, monkey supremacy, various religious cults.
%     \item (2020-22) Slacked severely in the third and fourth years of my undergraduate degree.
% \end{anticv}

% \vspace{0.5em}

% Those were different times, I was a different person, I was under the influence of intoxicating substances, severe and continued lapse in my judgement, etc.

\end{document}
