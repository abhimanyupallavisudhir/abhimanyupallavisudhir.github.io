\documentclass{article}

\usepackage{geometry, xcolor}
\usepackage{titling, titlesec, enumitem, natbib}
\usepackage{hyperref, cite, microtype, babel}
\usepackage{mdframed}
\usepackage[normalem]{ulem}

\setlength{\droptitle}{-10em}
\newcommand{\subtitle}[1]{
  \posttitle{
    \par\end{center}
    \begin{center}\normalsize#1\end{center}
    \vskip0.5em}}
\titleformat{\section}
  {\large\bfseries}{\thesection}{1em}{}
\titleformat{\subsection}
  {\normalsize\bfseries}{\thesubsection}{1em}{}
\titleformat{\subsubsection}
  {\normalsize\itshape}{\thesubsubsection}{1em}{}
\titlespacing*{\section}{0pt}{2ex}{1ex}
\titlespacing*{\subsection}{0pt}{1ex}{0.5ex}
\titlespacing*{\subsubsection}{0pt}{0.5ex}{0.25ex}
%\setlist{nosep}
%\setlength{\bibsep}{0.3em}
\setitemize{itemsep=0.25em,topsep=0.5em,parsep=0pt,partopsep=0pt}

\newenvironment{relatedwork}
   {
     \begin{mdframed}[
       leftmargin=1cm,
       rightmargin=0cm,
       innerleftmargin=10pt,
       innerrightmargin=0pt,
       innertopmargin=0.5em,
       innerbottommargin=0.5em,
       linewidth=1pt,
       linecolor=gray,
       topline=false,
       bottomline=false,
       rightline=false
     ]
     \footnotesize
   }
   {
     \end{mdframed}
   }

\newenvironment{anticv}
{ \begin{itemize}
    \setlength{\itemsep}{0.5em}
    \setlength{\parskip}{0pt}
    \setlength{\parsep}{0pt}     }
{ \end{itemize}                  } 
   

\newcommand{\disown}[1]{\sout{#1}}
\newcommand{\archive}{\color{lightgray}}


\title{\Large\bf Abhimanyu Pallavi Sudhir}
\subtitle{AI and game theory researcher. \\ %\href{mailto:abhimanyupallavisudhir@gmail.com}{abhimanyupallavisudhir@gmail.com} / +44-7771824896 / 55 Leinster Square, London -- W2 4PW 
\vspace{-2em}}
\predate{}\postdate{}\date{}
\preauthor{}\postauthor{}\author{}

\begin{document}

\begingroup
\let\center\flushleft
\let\endcenter\endflushleft
\maketitle
\endgroup

\section*{Formal education}
\begin{itemize}
    \item University of Warwick $\cdot$ PhD Computer Science $\cdot$ 2022-26 -- supervisor: Long-Tran-Thanh %; Bounded rationality and market dynamics
    \item Imperial College London $\cdot$ MSci Mathematics $\cdot$ 2018-22 -- 1st class honors
          {\archive
    \item Dhirubhai Ambani International School, Mumbai $\cdot$ IB Diploma $\cdot$ 2013-18 -- 44/45
    \item NUS High School, Singapore $\cdot$ 2012-13
    \item Bukit View Primary School, Singapore $\cdot$ 2006-11
          }
\end{itemize}

\section*{Internships}
\begin{itemize}
    \item Goldman Sachs $\cdot$ AI Research Intern $\cdot$ Jan-Aug 2021, London -- Developed and implemented novel methods in NLP and recurrent neural networks for financial forecasting
          {\archive
    \item Jane Street $\cdot$ Spring Week $\cdot$ 14-17 Apr 2020 -- [cancelled due to COVID-19 lockdowns]
    \item Schroders $\cdot$ Spring Week $\cdot$ 12-13 Aug 2020 -- [held virtually due to COVID-19 lockdowns]
    \item Jane Street $\cdot$ Fall Insight Day $\cdot$ 30 Oct 2018
          }
\end{itemize}

\section*{Research}

\subsection*{Markets and AI (PhD work)}

My current work focuses on building an algorithmic model of market dynamics to design AI agents with a market-based structure, and developing prediction market mechanisms to elicit beliefs about latent space variables to boost interpretability.

\begin{itemize}

    \item
          Abhimanyu Pallavi Sudhir and Long-Tran Thanh (2024), ``Betting on what is neither verifiable nor falsifiable'', \href{https://arxiv.org/abs/2402.14021}{arxiv.org/abs/2402.14021}

    \item
          Abhimanyu Pallavi Sudhir (2021),
          ``A mathematical definition of property rights in a Debreu economy'',
          \href{https://arxiv.org/abs/2107.09651}{arxiv.org/abs/2107.09651}

\end{itemize}

\begin{relatedwork}
    \emph{Related write-ups and talks.}
    \begin{itemize}[label=—]
        \item Blog posts
              \begin{itemize}
                  \item \href{https://www.lesswrong.com/posts/id84oe3LxdzoqinKA/betting-on-what-is-un-falsifiable-and-un-verifiable}{``Betting on what is un-falsifiable and un-verifiable'' (2023) on LessWrong}
                  \item \href{https://www.lesswrong.com/posts/xqxXrAohXSD3akYCg/meaningful-things-are-those-the-universe-possesses-a}{``Meaningful things are those the universe possesses a semantics for'' (2022) on LessWrong}
              \end{itemize}
        \item PhD formal reports and presentations
              \begin{itemize}
                  \item Year 1 Annual Report \href{https://abhimanyu.io/legacy_writing/PhD_reports/y1_annual_report.pdf}{abhimanyu.io/legacy\_writing/PhD\_reports/y1\_annual\_report.pdf}
                  \item PhD proposal \href{https://abhimanyu.io/legacy_writing/PhD_reports/y0_proposal.pdf}{abhimanyu.io/legacy\_writing/PhD\_reports/y0\_proposal.pdf}
              \end{itemize}
        \item Miscellaneous presentations
              \begin{itemize}
                  \item ``Mechanism design for AI alignment'' [poster at the Co-operative AI Foundation (CAIF) summer workshop, 2023]:

                        \href{https://abhimanyu.io/legacy_writing/PhD_presentations/caif.pdf}{abhimanyu.io/legacy\_writing/PhD\_presentations/caif.pdf}

                  \item ``Betting on what is neither verifiable nor falsifiable'' [Warwick Postgraduate colloquium (Dec 2023) \& Warwick Cake Talk (Nov 2023)]:

                        \href{https://abhimanyu.io/legacy_writing/PhD_presentations/betting_nonvf.pdf}{abhimanu.io/legacy\_writing/PhD\_presentations/betting\_nonvf.pdf}

                  \item ``Algorithmic information is at the root of all our problems'' [Warwick Postgraduate colloquium (Mar 2023)]:

                        \href{https://abhimanyu.io/legacy_writing/PhD_presentations/algorithmic_info.pdf}{abhimanyu.io/legacy\_writing/PhD\_presentations/algorithmic\_info.pdf}

                  \item ``Incompleteness theorems and firing philosophers'' [Warwick Cake Talk (Feb 2023)]:

                        \href{https://abhimanyu.io/legacy_writing/PhD_presentations/incompleteness.pdf}{abhimanyu.io/legacy\_writing/PhD\_presentations/incompleteness.pdf}

                  \item ``A mathematical definition of property rights'' [Imperial Undergraduate Colloquium (Feb 2022), Sheffield SIAM-IMA Applied Math Conference (Jul 2021)]:

                        \href{https://abhimanyu.io/legacy_writing/Imperial_presentations/property_rights.pdf}{abhimanyu.io/legacy\_writing/Imperial\_presentations/property\_rights.pdf}
              \end{itemize}
    \end{itemize}
\end{relatedwork}

\subsection*{AI Debate (2024)}

Ongoing collaboration with a team supervised by Arjun Panickssery and Nina Rimsky, on training AI debaters for various natural language tasks.

\subsection*{Consistency checks and forecasting (2024)}

Ongoing collaboration with a team supervised by Daniel Paleka to develop a consistency benchmark for LLM forecasters.

\begin{itemize}

    \item
          Abhimanyu Pallavi Sudhir, Alejandro Alvarez, Adam Shen, and Daniel Paleka (2024), ``Consistency Checks for Language Model Forecasters'' \emph{Workshop paper, accepted to:} Agentic Markets Workshop at ICML 2024; NextGenAISafety Workshop at ICML 2024; Oxford ELLIS Robust LLMs Workshop 2024
          %\href{https://openreview.net/forum?id=3so6NRQZfG}{Agentic Markets Workshop at ICML 2024}, \href{https://openreview.net/forum?id=lhgMq3JtFS}{NextGenAISafety Workshop at ICML 2024}, \href{https://openreview.net/forum?id=KFMdHPi4sk}{Oxford ELLIS Robust LLMs Workshop 2024}
          % \begin{itemize}
          %   \item Agentic Markets Workshop at ICML 2024
          %   \item NextGenAISafety Workshop at ICML 2024
          %   \item Oxford ELLIS Robust LLMs 2024 Workshop
          % \end{itemize}
\end{itemize}

{\archive

\subsection*{General mathematics (Undergraduate work and prior)}

\begin{itemize}

    \item
          Abhimanyu Pallavi Sudhir (2019),
          ``Infinitesimal translations and a multivariate Gr\"unwald-Letnikov calculus'', \href{https://arxiv.org/abs/1904.02710}{arxiv.org/abs/1904.02710}

    \item
          Abhimanyu Pallavi Sudhir (2018),
          ``The generalized Cauchy derivative as a principal value of the Gr\"unwald-Letnikov fractional derivative for divergent expansions,'' \href{https://arxiv.org/abs/1809.08051}{arxiv.org/abs/1809.08051}

    \item
          Abhimanyu Pallavi Sudhir (2019),
          ``Generalisations of the determinant to interdimensional transformations: a review,'' \href{https://arxiv.org/abs/1904.08097}{arxiv.org/abs/1904.08097}

    \item
          \disown{Abhimanyu Pallavi Sudhir (2014),
              ``On the Determinant-like function and the Vector Determinant,''
              \emph{Advances in Applied Clifford Algebras} (24-3: 805-807), \href{https://link.springer.com/article/10.1007/s00006-014-0455-3}{doi:10.1007/s00006-014-0455-3}}

    \item
          \disown{Abhimanyu Pallavi Sudhir (2014),
              ``On the Properties of the Determinant-like function,''
              (presented at International Conferences on Mathematical Sciences, Chennai, July 17-19, 2014).}

    \item
          \disown{Abhimanyu Pallavi Sudhir (2013),
              ``Defining the Determinant-like function for m by n matrices using the exterior algebra,''
              \emph{Advances in Applied Clifford Algebras} (23-4: 787-792),
              \href{https://link.springer.com/article/10.1007/s00006-013-0416-2}{doi:10.1007/s00006-013-0416-2}}

    \item
          \disown{Abhimanyu Pallavi Sudhir (2012),
              ``The Representation of Matrices in unit vector notation,''
              \emph{Journal of Mathematics Research} (4-4: 86-91),
              \href{https://dx.doi.org/10.5539/jmr.v4n4p86}{doi:10.5539/jmr.v4n4p86}}

    \item \disown{All of the crank stuff I posted to PhysicsForums as a kid}

\end{itemize}

\begin{relatedwork}
    {\archive \emph{Related write-ups and talks.}
        \begin{itemize}[label=—]
            \item Fractional calculus presentation [IMA Tomorrow's Mathematicians Today (Feb 2019), Imperial Undergraduate Colloquium (Nov 2018)]:

                  \href{https://abhimanyu.io/legacy_writing/Imperial_presentations/fractional_calculus.pdf}{abhimanyu.io/legacy\_writing/Imperial\_presentations/fractional\_calculus.pdf}
            \item Intel ISEF (May 2015) [received AMS Karl Menger Award]
        \end{itemize}}
\end{relatedwork}

}

\section*{Academic service}

\begin{itemize}

    \item \emph{Reviewer for NextGenAISafety Workshop at ICML 2024} $\cdot$ 2024

    \item \emph{Teaching Assistant for CS141: Functional Programming (Warwick)} $\cdot$ 2023

    \item \emph{Reviewer for Advances in Applied Clifford Algebras (Springer)} $\cdot$ 2020-present

\end{itemize}

\section*{Workshops and courses}

\begin{itemize}

    \item \emph{Co-operative AI Foundation} $\cdot$ Jul 2023 $\cdot$ workshop on AI and cooperative game theory

          {\archive

    \item \emph{Machine Learning and Applied Statistics} $\cdot$ Jul 2019 $\cdot$ summer course at Imperial College Business School; 7.5 ECTS, score: 97.5\%

          }

\end{itemize}


\section*{Other projects}

\subsection*{Costly (2024)}

Wrote the Python package \texttt{costly} for estimating costs and running times of complex LLM workflows/experiments/pipelines in advance before spending money, via simulations.

Project page: \href{https://github.com/abhimanyupallavisudhir/costly}{github.com/abhimanyupallavisudhir/costly}

Install: \texttt{pip install costly}

\subsection*{Equivariant learning (2021-22)}

Final-year MSci project with Professor Jeroen Lamb at Imperial College London exploring equivariant learning and causal DAGs.

Report: \href{https://abhimanyu.io/legacy_writing/Imperial_reports/m4r.pdf}{abhimanyu.io/legacy\_writing/Imperial\_reports/m4r.pdf}

\subsection*{Local normal forms of analytical maps near fixed points (2020)}

Second-year MSci project with Professor Davoud Cheraghi at Imperial College London.

Report: \href{https://abhimanyu.io/legacy_writing/Imperial_reports/m2r.pdf}{abhimanyu.io/legacy\_writing/Imperial\_reports/m2r.pdf}

Presentation: \href{https://abhimanyu.io/legacy_writing/Imperial_reports/m2r_presentation.pdf}{abhimanyu.io/legacy\_writing/Imperial\_reports/m2r\_presentation.pdf}

\subsection*{Lie theory (2019)}

Undergraduate research project with Professor Richard Thomas at Imperial College London on Lie groups and algebras.

Report: \href{https://abhimanyu.io/legacy_writing/Imperial_reports/urop.pdf}{abhimanyu.io/legacy\_writing/Imperial\_reports/urop.pdf}

Presentation: \href{https://abhimanyu.io/legacy_writing/Imperial_presentations/lie_theory.pdf}{abhimanyu.io/legacy\_writing/Imperial\_presentations/lie\_theory.pdf}

\begin{relatedwork}
    \emph{Related write-ups and talks.}
    \begin{itemize}[label=—]
        \item Warwick-Imperial Autumn Meeting (Mar 2022) [cancelled due to COVID-19 lockdowns]
        \item Imperial Undergraduate Colloquium (Oct 2019)
        \item Imperial 3-minute thesis competition (Oct 2019)
    \end{itemize}
\end{relatedwork}


\subsection*{Lean (2018-19)}

Computerized formal proving in Lean with Professor Kevin Buzzard at Imperial College London.

\begin{itemize}
    \item Wrote the \href{https://github.com/leanprover-community/mathlib4/blob/3a7e6bb77ec51d8009107923a4c071a9473ecc5c/Mathlib/Order/Filter/FilterProduct.lean}{FilterProduct.lean} and \href{https://github.com/leanprover-community/mathlib4/blob/3a7e6bb77ec51d8009107923a4c071a9473ecc5c/Mathlib/Data/Real/Hyperreal.lean}{Hyperreal.lean} modules for the Lean math library
    \item Imperial first-year project poster: \href{https://abhimanyu.io/legacy_writing/Imperial_reports/m1r.pdf}{abhimanyu.io/legacy\_writing/Imperial\_reports/m1r.pdf}
    \item Formalized the first-year ``Foundations of Analysis'' module exam Blog post:

          \href{https://xenaproject.wordpress.com/2019/05/06/m1f-imperial-undergraduates-and-lean/}{xenaproject.wordpress.com/2019/05/06/m1f-imperial-undergraduates-and-lean/}
\end{itemize}

{\archive
\subsection*{PhysicsOverflow (2014-15)}

Co-founded \href{https://physicsoverflow.org}{PhysicsOverflow}, a postgraduate-level physics Q\&A site and open peer review system. See \href{https://en.wikipedia.org/wiki/PhysicsOverflow}{en.wikipedia.org/wiki/PhysicsOverflow} for more details.

\begin{itemize}

    \item
          \archive{Abhimanyu Pallavi Sudhir and Rahel Knoepfel (2015),
              ``PhysicsOverflow: A postgraduate-level physics Q\&A site and open peer review system,''
              \emph{Asia-Pacific Physics Newsletter} (4-1: 53-55),
              \href{https://dx.doi.org/10.1142/S2251158X15000193}{doi:10.1142/S2251158X15000193}}

\end{itemize}

}

{\archive

\subsection*{\disown{The Mathematics and Physics Encyclopedia (2010-14)}}

\begin{itemize}
    \item \disown{\href{https://psiepsilon.wikia.com}{psiepsilon.wikia.com}}
    \item \disown{\href{https://psiepsilon.wordpress.com}{psiepsilon.wordpress.com}}
    \item \disown{\href{https://www.youtube.com/user/abhi99ps}{youtube.com/user/abhi99ps}}
\end{itemize}

}

{\archive

\subsection*{Awards}

\begin{itemize}
    \item Scholarships
          \begin{itemize}
              \item Warwick PhD (2022-26) -- departmental full scholarship
              \item ICBS Machine Learning Summer course (2019) -- departmental full scholarship
          \end{itemize}
    \item Conferences and science fairs
          \begin{itemize}
              \item IMA TMT, London (2019) -- among 4 shortlisted for GCHQ prize
              \item Intel ISEF, Pittsburgh (2015) -- AMS Karl Menger Award
              \item \disown{International Conference on Mathematical Sciences 2014 -- Best Paper Award}
              \item IRIS National Science Fair (2014) -- Gold; Amul Top 3; GUJCOST Merit Award
              \item IRIS National Science Fair (2013) -- Silver; Special Physics Prize
          \end{itemize}
    \item Problem-solving and olympiads
          \begin{itemize}
              \item Imperial Mathematics Competition (2019) -- nationwide finalist
              \item IIT Math Olympiad (2017) -- sixth place nationally in India
              \item Regional Mathematical Olympiad (2016) -- Merit
          \end{itemize}
    \item Kid competitions
          \begin{itemize}
              \item 2012 Bukit Panjang High School Mathematics and Science Challenge -- Team 1st
              \item 2012 American Mathematics Contest -- Certificate of Achievement
              \item 2012 Rio Tinto Science Contest -- High Dist
              \item 2011 Singapore Mathematical Olympiad Junior -- Honorable Mention
              \item 2011 Singapore Mathematical Olympiad for Primary Schools -- Gold
              \item 2011 Singapore and ASEAN Schools' Math Olympiad -- Gold
              \item 2011 Anglo-Chinese Young Whizzes' Challenge -- Gold; Team Round -- Team 2nd
              \item 2011 River Valley Math Comp -- Individual 1st; Team 1st; Team round -- 2nd; Platinum
              \item 2011 St. Andrew's Math and Science Comp -- Individual 1st; Team 1st; Team round -- 1st
              \item 2011 Mathematical Olympiad Talent Quest -- Bronze; Team Round -- Team 3rd
              \item 2011 Australian Mathematics Competition -- High Dist
              \item 2011 Rio Tinto Science Contest -- Credit
              \item 2011 UNSW ICAS -- Math/Sci/English (Dist) Computers (Credit)
              \item 2010 NUSHS Singapore Primary Science Olympiad -- Gold
              \item 2010 NUSHS National Math Olympiad of Singapore -- Bronze
              \item 2010 Anglo-Chinese Mathlympics -- Individual 3rd; Gold
              \item 2010 Anglo-Chinese Young Whizzes' Challenge -- Gold
              \item 2010 Singapore and ASEAN Schools' Math Olympiad -- Gold
              \item 2010 Australian Mathematics Competition -- Dist
              \item 2010 UNSW ICAS -- Math (HighDist) Science (Dist) English/Writing/Computers (Credit)
              \item 2009 UNSW ICAS -- Math (HighDist) Science (Dist) English (Credit)
              \item 2009 Australian Mathematics Competition (Dist)
              \item 2008 UNSW ICAS -- Math/Science/English (Dist)
              \item 2008 Australian Mathematics Competition (Credit)
          \end{itemize}
\end{itemize}

}

\section*{Links}

\begin{itemize}
    \item Email: \href{mailto:abhimanyupallavisudhir@gmail.com}{abhimanyupallavisudhir@gmail.com}
    \item Contact: \href{tel:+44-7771824896}{+44-7771824896}
    \item Website: \href{https://abhimanyu.io/}{abhimanyu.io}
    \item Blog: \href{https://thewindingnumber.blogspot.com/}{TheWindingNumber.blogspot.com}
    \item Google Scholar: \href{https://scholar.google.com/citations?user=lb38BjYAAAAJ&hl=en}{scholar.google.com/citations?user=lb38BjYAAAAJ}
    \item Github: \href{https://github.com/abhimanyupallavisudhir}{github.com/abhimanyupallavisudhir}
    \item LessWrong: \href{https://www.lesswrong.com/users/abhimanyu-pallavi-sudhir}{lesswrong.com/users/abhimanyu-pallavi-sudhir}
    \item StackExchange: \href{https://math.stackexchange.com/users/78451/abhimanyu-pallavi-sudhir}{math.stackexchange.com/users/78451/abhimanyu-pallavi-sudhir}
    \item {\archive PhysicsOverflow: \href{https://physicsoverflow.org/user/dimension10}{physicsoverflow.org/user/dimension10}}
\end{itemize}


\vspace{1em}

\textbf{Key:} Regular, {\archive Archived}, {\archive \disown{Disowned}}

\end{document}
