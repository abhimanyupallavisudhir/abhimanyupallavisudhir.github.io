\newcommand{\comment}[1]{}

\comment{ Re: introduction
Miscellaneous comments on the importance of complex dynamics to more general dynamical systems stuff 
Importance of Julia, Fatou sets to complex dynamics. 
Motivation for making cases based on value of lambda. 
* lambda is the derivative at 0, thus captures the local behaviour -- lambda < 1, > 1 describes shrinking, expansion i.e. attraction, repulsion. 
* When lambda = 0, this "locally linear" description does not give us enough -- analogies with higher derivatives capturing local behaviour near a critical point 
* lambda = 1 is a boundary case -- here, not everything will converge or diverge. But fortunately (at least with rational case) we find notions of "attraction vectors" and "basins" around them describing where and how things converge and diverge. Julia, Fatou sets have more exotic behaviour. 
Basic theorems (e.g. complex analysis theorems) used in proofs.
* Cauchy derivative estimate (Lemma 1.2')
Notations and conventions
* Stuff like f, lambda, p, r. Assuming fixed pt = 0.
}

\comment{ Re: tasks 
To do after combining:
* Standardize notation
* Standardize numbering
* Write intro
* Add any useful intuitive stuff, maybe language consistency and illustrations
* Exercises at end (maybe some kind of "combined" exercise?)
* References (inline references needed?), TOC and other any other bureaucratic sections
}