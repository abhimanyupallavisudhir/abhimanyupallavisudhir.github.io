\documentclass[../main.tex]{subfiles}
\newcommand\postcrit{P}
\newcommand\dist{\operatorname{dist}}
\newcommand\deriv[2]{{#1}'(#2)}
\begin{document}


% chapter 11-specific stuff starts here
\subsection{Siegel Discs and the Postcritical Closure}

\note{I gave up on the attempt to set up all the machinery for this section, this will just be the statements of the theorems in the final report.}

In the previous sections, we've shown that every attracting periodic orbit attracts a critical point (Theorem \ref{8:thm:orbcrit}), and that every parabolic fixed point contains a critical point in each of its basins (ah fuck, did we not cover theorem 10.15 in Milnor?). For an irrationally indifferent fixed point, there are again relations --- albeit less direct --- between the dynamics and the set of critical points.

\begin{dfn} The \emph{postcritical closure} $\postcrit$ of a map $f : \hat{\C} \to \hat{\C}$ is the topological closure of the strict forward orbits of the critical points of $f$:
\[
\postcrit = \postcrit(f) = \cl{\bigcup\limits_{n > 0} \iter{f}{n}(V)}
\]
in which $V = \{ z \in \hat{\C} \mid \deriv{f}{z} = 0\}$.
\end{dfn}

Equivalently, $\postcrit(f)$ is the smallest closed set that contains the critical values of $\iter{f}{n}$ for every $n > 0$. 

The main result of this section will be that $\postcrit(f)$ contains the attracting periodic orbits of $f$, as well as  the indifferent periodic orbits of $f$ which lie in the Julia set. In particular, 

The following lemma takes care of the exceptional case:

\begin{lem}
Let $f : \hat{\C} \to \hat{\C}$ be a rational map of degree $d \ge 2$. If $\abs{\postcrit(F)} < 3$, then $f$ is conjugate to $z \mapsto z^{\pm d}$.
\end{lem}
\begin{proof}
    By the Riemann-Hurwitz formula, $f$ has $2d-2$ critical points, counting with multiplicity. See first that it cannot have only one critical point, since (invoking the hypothesis that $d \ge 2$) the single critical point would then have multiplicity $2d-2 > d-1$, which is absurd.
    Furthermore, $f$ cannot have a single fixed point 
    After conjugating by a Möbius transformation we may ...
    
    
\end{proof}

\subsubsection{The Poincaré Metric and the Schwarz-Pick Lemma}

In this section we present an abridged account of some machinery.

\begin{dfn}[Riemannian metric]
(something something inner product on tangent spaces.
\end{dfn}

It turns out that every hyperbolic Riemann surface admits, up to a positive constant multiple, a unique Riemannian metric that is invariant under conformal automorphisms.

If we in addition impose the condition that

We shall depend on the existence and uniqueness of Poincaré metric in the following discussion, but we do not give details of the construction. Instead, we present a couple of examples: in the case where we are considering a Riemann surface which is a subset of $\C$, 

\begin{exl}[unit disc]
The Poincaré metric on $\D$ is given by
\[
ds = \frac{2\abs{dz}}{1-\abs{z}^2}
\]
\end{exl}

\begin{exl}[punctured disc]
The Poincaré metric on $\D\setminus \{0\}$ is given by
\[
ds = \frac{\abs{dz}}{\abs{z} \log(1/\abs{z})}
\]
\end{exl}



\begin{dfn}[Poincaré arclength]
The \emph{length} of a path $\gamma : [a, b] \to S$ in a hyperbolic Riemann surface $S$ equipped with its Poincaré metric, is defined to be
\[
\int_a^b \norm{\deriv{\gamma}{t}} dt
\]
\end{dfn}

This notion of length enables us in turn to speak of \emph{distances} between points on a Riemann surface:


\begin{lem}\label{11:lem:pre-schwarz-pick}
Let $S$ and $T$ be hyperbolic Riemann surfaces equipped with their respective Poincaré metrics, and let $f : S \to T$ be holomorphic. Then $\norm{\deriv{f}{z}} \le 1$ for all $z \in S$, with equality when and only when $f$ is a covering map.
\end{lem}

We do not present a proof of this lemma here. We do derive from it the following consequence:

\begin{lem}[Schwarz-Pick]
Let $S$ and $T$ be hyperbolic Riemann surfaces equipped with their respective Poincaré metrics, and let $f : S \to T$ be holomorphic. 
\begin{itemize}
    \item \emph{isometry.} $f$ is a conformal isomorphism and preserves Poincaré distances:
    \[
    \dist_T(f(z_1), f(z_2)) = \dist_S(z_1, z_2)
    \]
    for all $z_1, z_2 \in S$.
    \item \emph{covering.} $f$ is a nontrivial covering map, and
    \[
    \dist_T(f(z_1), f(z_2)) \le \dist_S(z_1, z_2)
    \]
    for all $z_1, z_2 \in S$.
    \item \emph{contraction.} $f$ strictly decreases all non-zero distances. Furthermore, for compact $K \subseteq S$, there exists constant $c < 1$ such that
    \[
    \dist_T(f(z_1), f(z_2)) \le c \dist_S(z_1, z_2)
    \]
\end{itemize}
\end{lem}
\begin{proof}

In all other cases $\norm{\deriv{f}{z}} < 1$ for all $z \in S$. Then we observe the following inequality: for $K \subseteq S$ compact and $z_1, z_2 \in K$, take $\gamma : [a, b] \to S$ a curve with length equal to $\dist_S(z_1, z_2)$, then after possibly choosing a larger compact $K' \supseteq K$ so that the image of $\gamma$ falls entirely within $K'$, we have
\begin{align*}
\dist_T(f(z_1), f(z_2))
&\le \int_a^b \norm{\deriv{(f \circ \gamma)}{t}} dt
&\by{definition of $\dist_T$}\\
&\le \int_a^b \norm{\deriv{f}{\gamma(t)}} \norm{\deriv{\gamma}{t}} dt
&\by{chain rule and sub-multiplicativity of operator norm}\\
&\le \max_{z \in K'} \norm{\deriv{f}{z}} \int_a^b \norm{\deriv{\gamma}{t}} dt &\\
&\le c \dist_S(z_1, z_2) &\by{by hypothesis on $\gamma$}
\end{align*}
in which $c = \max_{z \in K'} \norm{\deriv{f}{z}} < 1$ is a constant.
\end{proof}

\subsubsection{Expansion outside of the Postcritical Closure}


\begin{lem}\label{lem:julia-in-cl-union}
Let $f : \hat{\C} \to \hat{\C}$ be a rational map of degree $d \ge 2$ and suppose $\abs{\postcrit(f)} \ge 3$. Then the Julia set of $f$ is contained in the closure of $\bigcup_{n \ge 0} P_n$. In particular, for every $z$ in the Julia set,
\[
\dist_Q(z,P_n) \tendsto 0 \quad\text{as}\quad n \tendsto \infty
\]
\end{lem}
\begin{proof}
Equivalently it suffices to show that the complement $U = \hat{\C}\setminus\cl{\bigcup_{n \ge 0} P_n}$ is contained in the Fatou set.

Therefore the family $\{\iter{f}{n}\}$ defined on $U$ omits 
\end{proof}



\begin{thm}\label{11:thm:expansion-one}
Let $f : \hat{\C} \to \hat{\C}$ be a rational map of degree $d \ge 2$ and suppose $\abs{\postcrit(f)} \ge 3$. Then for all $z$ such that $f(z) \notin \postcrit(f)$ we have
\[
\norm{\deriv{f}{z}} \ge 1
\]
with respect to the Poincaré metric on $Q$.
\end{thm}
\begin{proof}

\end{proof}

\begin{thm}\label{11:thm:expansion-iter}
Let $f : \hat{\C} \to \hat{\C}$ be a rational map of degree $d \ge 2$ and suppose $\abs{\postcrit(f)} \ge 3$. Then for all $z$ in the Julia set whose forward orbit stays outside of $\postcrit(f)$ (i.e. $\iter{f}{n}(z) \notin \postcrit(f)$ for every $n \ge 0$), we have
\[
\norm{\deriv{(\iter{f}{n})}{z}} \tendsto \infty
\]
as $n \tendsto \infty$, with respect to the Poincaré metric on $Q$.
\end{thm}
\begin{proof}

\end{proof}

\begin{thm}
Let $f : \hat{\C} \to \hat{\C}$ be a rational map of degree $d \ge 2$. Then its postcritical closure $\postcrit$ contains all attracting fixed points, as well as all indifferent fixed points which lie in the Julia set.
\end{thm}
\begin{proof}
In the case where $\abs{\postcrit(f)} < 3$, $f$ is conjugate to $z \mapsto z^{\pm d}$, which has
\begin{itemize}
    \item either attracting fixed points $0$ and $\infty$ in the case of $z \mapsto z^d$, or no attracting fixed points (but an attracting periodic orbit $\{0, \infty\}$) in the case $z \mapsto z^{-d}$.
    \item no indifferent fixed points --- one might see this by computing the power series around any other fixed point to find that it has multiplier $d$ with $\abs{d} \ge 2$, and must therefore be repelling.
    \item postcritical closure $P = \{0, \infty\}$.
\end{itemize}
thus we see that the conclusion holds in this case. In all other cases $\abs{\postcrit(f)} \ge 3$, and the conclusion follows from the two preceding results: by Theorem \ref{11:thm:expansion-one} a fixed point which does not lie in $\postcrit(f)$ cannot be attracting; by Theorem \ref{11:thm:expansion-iter} a fixed point in the Julia set whose forward orbit is disjoint from $\postcrit(f)$ cannot be indifferent.
\end{proof}

\begin{rmk}
In particular, $\postcrit(f)$ contains every Cremer fixed point. Furthermore, for every Cremer fixed point $\fix$ there is at least once critical point which contains $\fix$ in the closure of its forward orbit: this follows from $\fix \in \postcrit(f)$ together with the fact that $f$ has at most finitely many critical points.
\end{rmk}

\begin{thm}
Let $f : \hat{\C} \to \hat{\C}$ be a rational map of degree $d \ge 2$. Then the boundary of every Siegel disc is contained in $\postcrit(f)$.
\end{thm}
\begin{proof}
In the case where $f$ is conjugate to $z \mapsto z^{\pm d}$, $f$ has no Siegel discs and the conclusion holds vacuously. Thus assuming $\abs{\postcrit(f)} \ge 3$:
\end{proof}


\end{document}