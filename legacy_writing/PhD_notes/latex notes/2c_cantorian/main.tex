\documentclass{article}
\usepackage{preamble}

\title{Thing}
\author{Abhimanyu Pallavi Sudhir}
\date{13 November 2022}

\begin{document}

\maketitle

\section{Introduction}

We want to build a general framework of ``a universe with agents that form beliefs about the universe'' -- this framework would yield as special cases: formal logic and incompleteness, boundedly Bayes-rational agents, belief and market dynamics. There are some obvious questions that arise when we try to ask that our framework allow agents that can have beliefs about arbitrary meaningful things in this universe (a rather abstracted notion of general intelligence):

\begin{itemize}
    \item {[the Cantorian question]} Isn't there an obvious cardinality issue? -- if an agent assigns beliefs to every possible state of the world, and the state includes every possible belief-assignment of the agent, then the state should be larger than at least its power set.
    \item {[the G\"odelian question]} If this framework can contain agents capable of executing any computation, then in particular it can contain an agent that acts contrary to the beliefs of an agent (either itself or another agent), thus limiting either the completeness or soundness of the latter agent (terms that will need to be defined more generally).
\end{itemize}




\end{document}
