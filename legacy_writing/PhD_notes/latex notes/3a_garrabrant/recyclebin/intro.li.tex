\subsection{Review of Logical induction}





Bounded rationality may be regarded as the problem of modeling rationality that accounts not only for the scarcity of \emph{statistical information} (as EUM does, and as as-if models have unsuccessfully attempted to model bounded rationality as doing \cite{rationalization_discounting, rationalization_inattention, rationalization_inattention_2}), but also for the scarcity of \emph{algorithmic information} (which is hard, due to the infinite regress of meta-reasoning).

The basic position taken in this paper is that a natural framework for modeling ``optimality conditional on algorithmic information'' is a \emph{market}. Markets represent beliefs (through prices) and inform decisions; unlike for perfectly rational agents, these beliefs are incomplete (not every state of the world has a priced asset); ``inconsistency'' of beliefs is arbitrage or non-equilibrium; there is a great generality to the assets a market may have, including assets that represent ``reflective'' beliefs, and markets respond to changes in the environment by creating new assets and pricing them; most importantly, with regards to the Efficient Market Hypothesis: perfect optimality is uncomputable \cite{chaitin_marketequilibrium, emh_nogo_1, emh_nogo_2, emh_nogo_3, emh_nogo_4, emh_nogo_5, emh_nogo_6, emh_nogo_7, emh_nogo_8, emh_nogo_9, emh_nogo_10, emh_nogo_11, emh_nogo_12, emh_nogo_13}, yet it may be argued that markets are efficient when ``accounting for the scarcity of algorithmic information'', whatever that would mean. There is also some work suggesting that markets are capable of reaching efficient outcomes even when individual traders are not very intelligent \cite{gs_1, gs_2, gs_3, gs_laibson, gs_schwarz} -- suggesting that intelligence could be modeled as a market of simple algorithms, not very carefully chosen. 

The single most relevant work to ours is \cite{logical_induction} 

\begin{definition}[Logical Inductor set-up]\label{def:li}
    Let $\mathcal{S}$ be a set of sentences in a language, and $\mathcal{T}$ be an enumerator of some sentences (called ``theorems'') in $\mathcal{S}$, called a ``deductive process''. 
    \begin{itemize}
        \item A \emph{trader} $T$ is a function $(\mathcal{S}\to[0,1])^n$
    \end{itemize}

\end{definition}