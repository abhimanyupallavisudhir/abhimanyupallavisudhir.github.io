\subsection{Review of work in bounded rationality}

The non-computability of Bayes-rational expected utility maximization (EUM) is not an obstacle to standard game-theoretic and economic paradigms that merely seek to demonstrate the existence of an equilibrium, but it does limit the usefulness of EUM for modeling AI and for studying game/market dynamics.

This has lead to several lines of research seeking to model bounded rationality. Early approaches included the explicit modeling of \emph{specific heuristics} (reviewed in \cite{oldreview_heuristics}), \emph{as-if theories} (reviewed and criticised for their lack of predictive usefulness in \cite{norationalization_friedman}) and explicit modeling of agents as limited cognitive architectures, usually as \emph{finite automata} (reviewed in \cite{oldreview_aumann}). 

More recent work has relied on a unspecified meta-reasoner to decide the optimality of a program or find an optimal program. For example: \emph{bounded optimality} \cite{bounded_optimality_LHS, bounded_optimality_RS, bounded_optimality_zilberstein}  defines optimality of a program with respect to computational constraints but does not provide a general way to obtain the boundedly optimal program; \emph{machine games} \cite{halpern_game, halpern_choice, tennenholtz} are games with programs for choices, and while these games have equilibria under weak assumptions, there is no guarantee that this equilibrium is reached, even asymptotically. A much more general theory is the \emph{thermodynamic formulation} \cite{thermo_info, thermo_thermo, thermo_main}, which appeals to the underlying physics of computation and formulates a utility function and a computational-cost-adjusted utility function in thermodynamic terms.

We are left with the impression that despite these advancements, something is lacking in our understanding of bounded rationality. 

If all we wanted was a \emph{descriptive} theory, we could simply posit any model of computation as a ``theory of bounded rationality'', since in principle, an agent could be any program. Economics would then simply be the theory of distributed computing. But that would not help us answer any \emph{normative} questions, such as what a program ``truly'' wants, i.e. what its utility function would be if it had all the (algorithmic) information in the world, or results about welfare and normative economics. Other desiderata we may seek for a theory of bounded rationality include: a framework for assessing when an algorithm is \emph{asymptotically bounded optimal} (a term defined in \cite{bounded_optimality_intelligence}, and argued therein to capture the concept of \emph{intelligence}); that it should act as a theory of logical uncertainty, or that it should tend toward Bayes rationality in some natural limit (e.g. unlimited computational power). There is also the neoclassical demand \cite{oldreview_arrow} that the theory have a Bayesian or rational interpretation, and the contradicting heuristic or behavioural demand \cite{oldreview_gigerenzer} that it imitate real mechanisms by which agents may act -- these demands are best ignored, as the first is impossible, and the second seems to ask for a lower level of abstraction than a true theory of bounded rationality.

At this point, it is reasonable to critique that perhaps descriptive usefulness and normative usefulness are orthogonal goals that cannot be achieved by a single theory; that there \emph{is} no general theory of bounded rationality to expect. This criticism can only be conclusively refuted by actually constructing a framework that does achieve these goals.