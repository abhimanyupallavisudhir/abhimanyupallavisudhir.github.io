\documentclass[smallextended]{svjour3}

\usepackage[utf8]{inputenc}
\usepackage{amsmath, amssymb, amsfonts}
\usepackage{hyperref, url}

\title{``Observable utility'' as a generalization of GDP outside the consumer-firm model}
\author{Abhimanyu Pallavi Sudhir}
\institute{Department of Mathematics, Imperial College, London, United Kingdom. \\\email{ap6218@ic.ac.uk} \\ORCID: \href{https://orcid.org/0000-0002-2506-0515}{0000-0002-2506-0515}}
\date{\vspace{-5em}}

\begin{document}

\maketitle

\begin{abstract}
    The Gross Domestic Product (GDP) is a popular proxy for aggregate welfare in many practical settings. However, a theoretical inadequacy of GDP is that its definition rests on the notion of a ``final good'', and as a result is only well-defined within what we call the consumer-firm model of economics. In this paper, we proposed a new metric of welfare, that we term ``observable utility'', which is defined in a more general context. We prove some basic results about the validity of this metric, and demonstrate that it acts as a natural generalization of GDP, noting the conditions under which it reduces to the latter.
\end{abstract}

\section{Declarations}

\subsection{Funding}
Not applicable.

\subsection{Competing interests}
Not applicable.

\subsection{Availability of data and material}
Not applicable.

\subsection{Code availability}
Not applicable.

\end{document}